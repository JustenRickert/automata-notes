% Created 2017-05-08 Mon 16:22
% Intended LaTeX compiler: pdflatex
\documentclass[12pt]{article}
\usepackage[utf8]{inputenc}
\usepackage[T1]{fontenc}
\usepackage{graphicx}
\usepackage{grffile}
\usepackage{longtable}
\usepackage{wrapfig}
\usepackage{rotating}
\usepackage[normalem]{ulem}
\usepackage{amsmath}
\usepackage{textcomp}
\usepackage{amssymb}
\usepackage{capt-of}
\usepackage{hyperref}
\usepackage[margin=1.0in]{geometry}
\documentclass{article}
\usepackage{setspace,mathrsfs,amsmath,amsthm,amssymb,graphicx,cancel,lmodern,mathtools}
\author{Justen Rickert}
\date{\today}
\title{}
\hypersetup{
 pdfauthor={Justen Rickert},
 pdftitle={},
 pdfkeywords={},
 pdfsubject={},
 pdfcreator={Emacs 26.0.50.1 (Org mode 9.0.5)}, 
 pdflang={English}}
\begin{document}

\newcommand\abs[1]{\left|#1\right|}
\newcommand\deg{\textdegree}
\newcommand\Real{\mathbb{R}}
\newcommand\Natural{\mathbb{N}}
\newcommand\sube{\subseteq}
\newcommand\supe{\supseteq}
\newcommand\sub{\subset}
\newcommand\sup{\supset}
\newcommand\setm{\setminus}
\newcommand\pr{\ensuremath{'}}
\newcommand\R{\mathcal{R}}
\newcommand\calR{\mathcal{R}}
\newcommand\calP{\mathcal{P}}
\newcommand\pow{\mathscr{P}}
\newcommand\indX{\mathscr{X}}
\newcommand\nil{\varnothing}

\newtheorem{lemma}[Theorem]{lemma}
\theoremstyle{definition}
\newtheorem{definition}{Def}[section]

\section{Chapter One}
\label{sec:org39ce2ba}
\subsection{Automaton}
\label{sec:org9cc5937}
\begin{definition}[Finite Automaton]
  A \textit{\textbf{finite automaton}} is a 5-tuple $(Q,\Sigma,\delta,q_0,F)$,
  where:
  \begin{enumerate}
  \item $Q$ is a finite set called the \textit{\textbf{states}},
  \item $\Sigma$ is a finite set called the \textit{\textbf{alphabet}},
  \item $\delta$: $Q\times\Sigma\longrightarrow Q$ is the
    $\textit{\textbf{transition function}}$,
  \item $q_0\in Q$ is the \textit{\textbf{start state}}, and
  \item $F\subseteq Q$ is the \textit{\textbf{set of accept states}}.
  \end{enumerate}
\end{definition}

\begin{definition}[Nondeterministic Finite Automaton]
  A \textit{\textbf{nondeterministic finite automaton}} is a 5-tuple
  $(Q,\Sigma,\delta,q_0,F)$, where:
  \begin{enumerate}
  \item $Q$ is a finite set called the \textit{\textbf{states}},
  \item $\Sigma$ is a finite set called the \textit{\textbf{alphabet}},
  \item $\delta$: $Q\times\Sigma\longrightarrow \pow(Q)$ is the
    $\textit{\textbf{transition function}}$,
  \item $q_0\in Q$ is the \textit{\textbf{start state}}, and
  \item $F\subseteq Q$ is the \textit{\textbf{set of accept states}}.
  \end{enumerate}
\end{definition}

\begin{definition}[Generalized Nondeterministic Finite Automaton]
  A \textit{\textbf{generalized nondeterministic finite automaton}} is a
  5-tuple $(Q,\Sigma,\delta,q_0,F)$, where:
  \begin{enumerate}
  \item $Q$ is a finite set called the \textit{\textbf{states}},
  \item $\Sigma$ is the \textit{\textbf{input alphabet}},
  \item $\delta : (Q-\{q_{accept}\}) \times (Q-\{q_{start}\}) \longrightarrow
    \R$ is the $\textit{\textbf{transition function}}$,
  \item $q_{start}$ is the \textit{\textbf{start state}}, and
  \item $q_{accept}$ is the \textit{\textbf{accept state}}.
  \end{enumerate}
  The symbol $\R$ is the collection of all regular expressions over the
  alphabet $\Sigma$, and $q_{start}$ and $q_{accept}$ are the start and
  accept states. If $\delta(q_i,q_j)=R$, the arrow from state $q_i$ to state
  $q_j$ has the regular expression $R$ as its label. The domain of the
  transitition function is $(Q-\{q_{accept}\})\times(Q-\{q_{start}\})$
  because an arrow connects every state to every other state, except that no
  arrows are coming from $q_{accept}$ or going to $q_{start}$.
\end{definition}

\subsection{Computation}
\label{sec:orgf13c091}
\begin{definition}[Computation]
  Let $M=(Q,\Sigma,\delta,q_0,F)$ be a finite automaton and let $w=w_1w_2\dots
  w_n$ be a string where each $w_i$ is a member of the alphabet $\Sigma$. Then
  $M$ accepts $w$ if a sequence of states $r_0,r_1,\dots,r_n\in Q$ exists with
  three conditions:
  \begin{enumerate}
  \item $r_{0}=q_{0}$,
  \item $\delta(r_i,w_{i+1})=r_{i+1}$, for $i=0,\dots,n-1$, and
  \item $r_n\in F$.
  \end{enumerate}
\end{definition}

\begin{definition}[recognizes]
  $M$ \textit{\textbf{recognizes language}} $A$ if $A=\{w : M \text{ accepts }
  w\}$.
\end{definition}

\subsection{Regular}
\label{sec:orgae52da3}
\begin{definition}[Regular]
  A language is called a \textit{\textbf{regular language}} if some finite
  automaton recognizes it.
\end{definition}

\begin{definition}[Regular Operations]
  Let $A$ and $B$ be languages. We define the regular operations
  \textit{\textbf{union}}, \textit{\textbf{concatenation}} and
  \textit{\textbf{star}} as follows:
  \begin{itemize}
    \item \textbf{Union}: $A \cup B=\{x : x\in A \text{ or } x\in B\}$.
    \item \textbf{Concatenation}: $A \circ B=\{xy : x\in A \text{ and } y\in
    B\}$.
    \item \textbf{Star}: $A^*=\{x_1x_2\dots x_k : k\ge0 \text{ and each }
    x_i\in A\}$
  \end{itemize}
  In arithmetic, we say that $\times$ has precedence over over + to mean that
  when there is a choice, we do the $\times$ operation first. Thus in
  $2+3\times4$, the $3\times4$ is done before the addition. To have the
  addition done first, we must add paretheses to to obtain $(2+3)\times4$. In
  regular expressions, the star operation is done first, followed by
  concatenation, and finally union, unless parentheses change the usual
  order.
\end{definition}

\begin{definition}[Regular Operations]
  Say that $R$ is a {\textit{\textbfregular expression}} if $R$ is:
  \begin{enumerate}
    \item $a$ for some $A$ in the alphabet $\Sigma$,
    \item $\varepsilon$,
    \item $\varnothing$,
    \item $(R_1 \cup R_2)$, where $R_1$ and $R_2$ are regular expressions,
    \item $(R_1 \circ R_2)$, where $R_1$ and $R_2$ are regular expressions, or
    \item $(R_1^*)$, where $R_1$ is a regular expression.
  \end{enumerate}
  In items 1 and 2, the regular expressions $A$ and $\varepsilon$ represent the
  languages $\{a\}$ and $\{\varepsilon\}$, respectively. In item 3, the regular
  expression $\varnothing$ represents the empty language. In items 4, 5, and 6,
  the expressions represent the languages obtained by taking the union or
  concatenation of the languages $R_1$ and $R_2$, or the star of the language
  $R_1$, respectively.
\end{definition}

\begin{definition}[Equivalence With Finite Automata]
  \label{Equivalence With Finite Automata}
  A language is \textit{\textbf{regular}} if and only if some regular expression
  describes it.
\end{definition}

\begin{definition}{Equivalence With Regular Expressions}
  \textbf{Def \ref{Equivalence With Finite Automata}} means as well that if
  a language is described by a $\textit{\textbf{regular expression}}$, then
  it is $\textit{\textbf{regular}}$.
\end{definition}

\subsection{Nonregular Languages}
\label{sec:org6584c3b}
\begin{definition}[Puming Lemma]
  If $A$  is a  regular language,  then there  is a  number $p$  (the pumping
  length) where if  s is any string in  $A$ of length at least  $p$, then $s$
  may  be  divided  into  three pieces,  $s=xyz$,  satisfying  the  following
  conditions:
  \begin{enumerate}
    \item for each $i \ge 0$, $xy^iz \in A$,
    \item $\abs{y} > 0$, and 
    \item $\abs{xy} \le p$.
  \end{enumerate}
\end{definition}

\section{Chapter Two}
\label{sec:org6fb53e9}
\subsection{Context-Free Grammar}
\label{sec:org2e2f0d1}
\begin{definition}[Context-Free Grammar]
  A \textit{\textbf{context-free grammar}} is a 4-tuple $(V,\Sigma,R,S)$,
  where :---
  \begin{enumerate}
  \item $V$ is a finite set called the \textit{\textbf{variable}},
  \item $\Sigma$ is a finite set, disjoint from $V$, called the
    \textit{\textbf{terminals}},
  \item $R$ is a finite set of \textit{\textbf{rules}}, with each rule being a
    variable and a string of variables and terminals, and
  \item $S \in V$ is the start variable.
  \end{enumerate}
  The \textit{\textbf{language of the grammar}} is $\{w\in\Sigma^* : S
  \xRightarrow[]{\text{*}} w\}$. That is, say that $u$
  \textit{\textbf{derives}} $v$, written $u \xRightarrow[]{\text{*}} v$, if
  $u=v$ or if a sequence $u_1,u_2,\dots,u_k$ exists for $k\ge0$ and
  $$u\Rightarrow u_1 \Rightarrow u_2 \Rightarrow \dots \Rightarrow u_k
  \Rightarrow v.$$
\end{definition}

\begin{definition}[Chomsky Normal Form]
  A context-free grammar is in Chomsky normal form if every rule is of the
  form :---
  \begin{center}
    \begin{tabular}{l}
      $A \rightarrow BC$ \\
      $A \rightarrow a$ \\
    \end{tabular}
  \end{center}
  where $a$ is any terminal and $A$, $B$, and $C$ are any variable---except
  that $B$ and $C$ may not be the start variable. In addition, we permit the
  Rule $S \rightarrow \varepsilon$, where $S$ is the start variable.
\end{definition}

\subsection{Pushdown Automata}
\label{sec:orgdcf66e8}
\begin{definition}[Pushdown Automaton]
  A \textit{\textbf{pusdown automaton}} is a 6-tuple
  $(Q,\Sigma,\Gamma,\delta,q_0,F)$, where $Q$, $\Sigma$, $\Gamma$, and $F$
  are all finite sets, and :---
  \begin{enumerate}
  \item $Q$ is the set of states,
  \item $\Sigma$ is the input alphabet,
  \item $\Gamma$ is the stack alphabet,
  \item $\delta : Q\times\Sigma_\varepsilon\times\Gamma_\varepsilon
    \longrightarrow \calP(Q\times\Gamma_\varepsilon)$ is the transition function,
  \item $q_0 \in Q$ is the start state, and
  \item $F \sube Q$ is the set of accept states.
  \end{enumerate}
  Recall that $\Sigma_\varepsilon=\Sigma\cup\{\varepsilon\}$ and
  $\Gamma_\varepsilon=\Gamma\cup\{\varepsilon\}$. The domain of the transition
  function is $Q\times\Sigma_\varepsilon\times\Gamma_\varepsilon$. Thus the
  current state, next input symbol read, and top symbol of the stack determine
  the next move of a pushdown automaton.
\end{definition}

\begin{definition}[Equivalence With Context-Free Grammar]
  A language is context free if and only if some pushdown automaton recognizes
  it. If a language is context free, then some pushdown automaton recognizes
  it.
\end{definition}

\section{Chapter Three}
\label{sec:orgeec4b9d}
\subsection{Turing Machine}
\label{sec:orgaeaab3d}
\begin{definition}[Turing Machine]
  A  \textit{\textbf{Turing  machine}}  is  a  7-tuple  $(Q,  \Sigma,  \Gamma,  \delta,  q_0,
  q_{accepts}, q_{reject})$, where $Q, \Sigma, \Gamma$ are all finite sets and
  \begin{enumerate}
  \item $Q$ is the set of states,
  \item $\Sigma$ is the input alphabet not containing the blank symbol $\_$,
  \item $\Gamma$ is the tap alphabet, where $\_ \in \Gamma$ and $\Sigma \sube \Gamma$,
  \item $\delta : Q \times \Gamma \longrightarrow Q \times  \Gamma \times \{ $L$, $R$ \}$ is the transition
    function,
  \item $q_0 \in Q$ is the start state,
  \item $q_{accept} \in Q$ is the accept state, and
  \item $q_{reject} \in Q$ is the reject state, where $q_{reject} \ne q_{accept}$.
  \end{enumerate}
\end{definition}

\begin{definition}[Multitape Turing Machine]
  The  only difference  is that  $$\delta :  Q  \times \Gamma^k  \longrightarrow Q  \times \Gamma^k  \times
  \{\text{L}, \text{R}, \text{S}\}^k,$$ where $k$ is the number of tapes. The
  expression  $$\delta(q_i,  a_1  ...,  a_k)  = (q_j,  b_1,  ...,  b_k,  \text{L},
  \text{R}, ..., \text{L})$$ means that if  the machine is in state $q_i$ and
  heads $1$ through $k$ are reading  symbols $a_1$ through $a_k$, the machine
  goes to state  $q_j$, writes symbols $b_1$ through $b_k$,  and directs each
  head to mave left or right, or to stay put, as specified.
\end{definition}


\begin{definition}[Nondeterministic Turing Machine]
  The  only difference  is that  $$\delta :  Q \times  \Gamma \longrightarrow  \pow(Q \times  \Gamma \times
  \{\text{L},  \text{R}\}).$$  All  of  the  nondeterministic  paths  can  be
  simulated on a multitape Turing Machine.
\end{definition}

\phantomsection
\label{orgf2f8120}
\begin{itemize}
\item \emph{\textbf{start configuration}}
\item \emph{\textbf{accepting configuration}}
\item \emph{\textbf{rejecting configuration}}
\item \emph{\textbf{halting configuration}}
\end{itemize}

\phantomsection
\label{org6b51766}
\begin{itemize}
\item Call a language \emph{\textbf{Turing-recognizable}} if some Turing machine recognizes
it. (\emph{\textbf{recursively enumerable language}})
\begin{itemize}
\item A language is Turing-recognizable \(\iff\) some enumerator enumerates it.
\end{itemize}

\item Call a language \emph{\textbf{Turing-decidable}} or simply decidable if some Turing
machine decides it. (\emph{\textbf{recursive language}})
\begin{itemize}
\item A language is decidable \(\iff\) some nondeterministic Turing machine
decides it.
\end{itemize}
\end{itemize}

\section{Chapter Four}
\label{sec:orgf72244e}
\subsection{Decidable}
\label{sec:orgd05dfc7}
\begin{itemize}
\item \(A_{DFA}=\{\langle B,w \rangle : B \text{ is a DFA that accepts input string }
   w\}\) is a \uline{decidable language}.
\begin{itemize}
\item \(M=\) "On input \(\lang B,w \rang\):
\begin{enumerate}
\item Simulate \(B\) on input \(w\).
\item If the simulation ends in an accept state, \emph{accept}. If it ends in a
nonaccepting state, \emph{reject}"
\end{enumerate}
\end{itemize}

\item \(A_{NFA}=\{\langle{}B,w\rangle : B \text{ is an NFA that accepts input
   string } w\}\) is a \uline{decidable language}.
\begin{itemize}
\item \(N=\) "On input \(\langle B,w \rangle\):
\begin{enumerate}
\item Convert NFA \(B\) to an equivalent DFA \(C\), using the procedure for this
conversion given in Theorem 1.39.
\item Run TM \(M\) from Theorem 4.1 on input \(\langle C,w \rangle\).
\item If \(M\) accepts, \emph{accept}; otherwise, \emph{reject}."
\end{enumerate}
\end{itemize}

\item \(A_{REX}=\{\langle{}R,w\rangle : R \text{ is a regular expression that
   generates string } w\}\) is a \uline{decidable language}.
\begin{itemize}
\item \(P=\) "On input \(\langle R,w \rangle\):
\begin{enumerate}
\item Convert regular expression \(R\) to an equivalent NFA \(A\) be using the
procedure for this conversion given in Theorem 1.54.
\item Run TM \(N\) on input \(\langle{}A,w\rangle\).
\item If \(N\) accepts, \emph{accept}; if \(N\) rejects, \emph{reject}."
\end{enumerate}
\end{itemize}

\item \(E_{DFA}=\{\langle A\rangle : A \text{ is a DFA and } L(A)=\varnothing\}\)
is a \uline{decidable language}.
\begin{itemize}
\item \(T=\) "On input \(\langle A \rangle\):
\begin{enumerate}
\item Mark the start state of \(A\).
\item Repeat until no new states get marked:
\begin{enumerate}
\item Mark any state that has a transition coming into it from any state that
is already marked.
\end{enumerate}
\item If no accept state is marked, \emph{accept}; otherwise, \emph{reject}."
\end{enumerate}
\end{itemize}

\item \(EQ_{DFA}=\{\langle A,B \rangle : A \text{ and } B \text{ are DFAs and }
   L(A)=L(B)\}\) is a \uline{decidable language}.
\begin{itemize}
\item This new DFA \(C\) accepts only those strings that are accepted by either \(A\)
or \(B\) but not by both. Thus, if \(A\) and \(B\) recognize the same language,
\(C\) will accept nothing.
$$L(C)=\left(L(A) \cap \overline{L(B)}\right) \cup
     \left(\overline{L(A)} \cap L(B)\right)$$
\item \(F=\) "On input \(\langle{}A,B\rangle\):
\begin{enumerate}
\item Construct DFA \(C\) as described.
\item Run TM \(T\) from Theorem 4.4 on input \(\langle{}C\rangle\).
\item If \(T\) accepts, \emph{accept}. If \(T\) rejects, \emph{reject}."
\end{enumerate}
\end{itemize}

\item \(A_{CFG}=\{\langle{}G,w\rangle : G \text{ is a CFG that generates string }
   w\}\) is a \uline{decidable language}.

\item \(E_{CFG}=\{\langle{}G\rangle : G \text{ is a CFG and } L(G)=\varnothing\}\)
is a \uline{decidable language}.
\end{itemize}

\subsection{Undecidable}
\label{sec:orgba64b64}
\begin{itemize}
\item \(A_{TM}=\{\langle{}M,w\rangle : M \text{ is a TM and } M \text{ accepts }
   w\}\) is an \uline{undecidable language}. It is also \uline{Turing-recognizable}. recognizers
\emph{are} more powerful than deciders. \(U\) recognizes \(A_{TM}\), it is the universal
Turing machine capable of simulating every other Turing machine. 
\begin{itemize}
\item \(U=\) "On input \(\langle{}M,w\rangle\):
\begin{enumerate}
\item Simulate \(M\) on input \(w\).
\item If \(M\) ever enters its accept state, \emph{accept}; if \(M\) ever enters its
reject state, \emph{rejects}."
\end{enumerate}
\item The proof of this relies on a technique called \emph{diagonalization}.
\end{itemize}

\item \(HALT_{TM}=\{\langle M,w \rangle : M \text{ is a TM and } M \text{ accepts } w\}\) is an
\uline{undecidable language}. It is also \uline{Turing-recognizable}. Let's assume for
the purpose of obtaining a contradiction that TM \(R\) decides /HALT/\(_{\text{TM}}\)
\begin{itemize}
\item \(U=\) "On input \(\langle M, w \rangle\), an encoding of a TM \(M\) and
a string \(w\):
\begin{enumerate}
\item Run TM \(R\) on input \(\langle M,w \rangle\).
\item If \(R\) rejects, \emph{reject}.
\item If \(R\) accepts, simulate \(M\) on \(w\) until it halts.
\item If \(M\) has accepted, \emph{accept}; if \(M\) has rejected, \emph{reject}."
\end{enumerate}
\item Clearly, if \(R\) decides HALT\(_{\text{TM}}\), then \(S\) decides A\(_{\text{TM}}\). Because A\(_{\text{TM}}\).
Because A\(_{\text{TM}}\) is undecidable, HALT\(_{\text{TM}}\) also must be undecidable.
\end{itemize}
\end{itemize}

\section{Chapter Five}
\label{sec:orgdedf500}
\subsection{Reducible}
\label{sec:org7620c09}
\begin{definition}[Computation History]
  Let $M$  be a Turing  Machine and $w$  an input string.  An \textbf{accepting
    computation history} for $M$ on $w$  is a sequence of configurations, $C_1,
    C_2, ..., C_l$, where $C_1$ is the start configuration of $M$ on $w$, $C_l$
    is  an  accepting configuration  of  $M$  on  $w$,  $C_l$ is  an  accepting
    configuration  of $M$,  and  each $C_i$  legally follows  from  $C_{i -  1}
    $according to  the rules of  $M$. A \textbf{rejecting  computation history}
    for  $M$ on  $w$ is  defined similarly,  except that  $C_l$ is  a rejecting
    configuration.

\end{definition}

\begin{definition}[Linear Bounded Automaton]
  A \textbf{linear  bounded automaton}  is a restricted  type of  Turing machine
  wherein the  tape head  isn't permitted to  move off the  portion of  the tape
  containing the input. If the machine tries  to move its head off either end of
  the input, the head stays where it is---in the same way that the head will not
  move off the left-hand end of an ordinary Turing machine's tape.
\end{definition}

\begin{lemma}
  Let $M$ be an LBA with $q$ states  and $g$ symbols in the tape alphabet. There
  are exactly $qng^n$  distinct configurations of $M$ for a  tape of length $n$.
  \end{lemma} 

\begin{itemize}
\item \(A_{LBA}\) is  \uline{decidable}. \(E_{LBA}\) is  \uline{undecidable}. \(E_{LBA} = \{  \langle M
   \rangle \mid M\) is an LBA and \(L(M) = \varnothing \}\)

\item ALL\(_{\text{CFG}}\) is \uline{undecidable}. \(ALL_{CFG} = \{ \langle G \rangle \mid G\) is a CFG
and \(L(G) = \Sigma^* \}\)
\end{itemize}

\begin{definition}[Computable Function]
  A function  $f : \Sigma^*  \longrightarrow \Sigma^*$ is  a \textbf{computable
  function} if  some Turing machine  $M$, on every  input $w$, halts  with just
  $f(w)$ on its tape.
\end{definition}

\begin{definition}[Mapping Reducible, Reduction]
  Language $A$ is \textbf{mapping reducible}  to language $B$, written $A \le_m
  B$,  if  there  is  a  computable  function  $f  :  \Sigma^*  \longrightarrow
  \Sigma^*$, where for every $w$, $$w \in A \iff f(w) \in B.$$ The function $f$
  is called the \textbf{reduction} from $A$ to $B$.
\end{definition}
\end{document}